\chapter{Литературный обзор}%
\label{cha:Литературный обзор}
\section{Гексагональный нитрид бора}%
\label{sec:Гексагональный нитрид бора}

Гексагональный нитрида бора (\ce{h-BN}) по своей структуре аналог графита, поэтому
иногда называют "белым графитом". Кристаллическая структура \ce{h-BN} представляет
из себя набор графеноподобных слоев, расположенных, в отличие от структуры 
графита точно один под другим с чередованием атомов B и N по оси Z. Расстояние 
между слоями в решетке нитрида бора равно 3,34 А, которое меньше, чем у графита
(3,40 А), что доказывает о более прочной связи между слоями в структуре BN.

Температура плавления h-BN под давлением азота больше \SI{3000}{\degreeCelsius}, плотность 
составляет \SI{2.3}{\gram\per\cm^3} Гексагональный нитрид бора обладает 
полупроводниковыми свойствами
(с шириной запрещенной зоны около 3,7 эВ), так же обладает люминесцентными 
свойствами при наличии небольшого количества примесей. Нитрид бора при 
комнатной температуре химически инертен, он практически не реагирует с 
кислородом или хлором, кислотами или щелочами. Подвергается действию кислорода 
и хлора при температурах выше \SI{700}{\degreeCelsius}. 
Вступает в реакцию с фтором (образуя \ce{BF3} 
и \ce{N2}) и с HF (образуя \ce{NH4BF4}); горячие растворы щелочей разлагают его с 
выделением \ce{NH3}.

Нитрид бора обладает превосходными антифрикционными свойствами. Нанопорошки
а так же покрытия нитрида бора широко используются в качестве сухой смазки \cite[]{_ru__2011}.
Данная особенность связана с сильной ковалентной связью между B - N базовой плоскости 
связанных между собой слабой Ван-дер-Ваальсой связью.

\subsection{Методы получения гексагонального нитрида бора}%
\label{sub:Методы получения гексагонального нитрида бора}


Широко распространены следующие методы синтеза гексагонального нитрида бора:

\begin{itemize}
    \item Плазмотермический
    \item Метод СВС
    \item Методы химического осаждения
    \item Методы прямого азотирования 
\end{itemize}

В самом общем случае синтез нитрида бора происходит по простой реакции \ce{B + N2}, 
которая может быть осуществлена в трубчатой печи сопротивления. Однако энергия, которую 
необходимо затратить на активацию молекулы \ce{N2} достаточно высока,
а технологические требования к морфологии и структуре материала ставят
задачи по поиску новых, оптимальных методик по синтезу различных структур. 

\subsection{Прямое азотирование}%
\label{sub:Прямое азотирование}

Первые исследовательские работы по изучению синтеза гексагонального BN учитывали
необходимость замены реагентов на прекурсоры с  меньшей энергией активации для 
прохождения реакции. Одним из решений может являться
замена \ce{N2} на \ce{NH4}, а чистого аморфного бора на \ce{H3BO3}. Первые
работы в области синтеза BN структур показали образование различных переходных 
продуктов: \ce{(B2O3)nNH} при \SI{200}{\degreeCelsius},
\ce{(BN)x(B2O3)y(NH3)z} при \SI{350}{\degreeCelsius}
\cite[]{economy_boron_1967}. Образование оксида бора с точкой плавления
\SI{450}{\degreeCelsius} приводит уменьшению удельной поверхности при
температурах выше температуры плавления, что приводит к агломерации
капель оксида бора и уменьшению контактной поверхности между реакционным
газом и прекурсором. Подобный эффект негативно сказывается на диффузии реакционного
газа через прекурсор, в результате чего в продуктах реакции наблюдается большое 
количество непрореагировавших реагентов, а так же полупродуктов. Чтобы преодолеть
данное технологическое ограничение синтез проводился в несколько этапов: синтез
в протоке реакционного газа прерывался, полупродукт перемалывался, далее синтез
возобновляли. Так же существенным технологическим осложнением являлось 
использование аммиака в качестве источника атомарного азота.

Для решения данной проблемы обычно используют инертные наполнители или
продукты реакции, которые не реагируют с исходными компонентами и газовой
средой, а так же не дают мельчайшим каплям оксида бора сливаться. И тут
возникает некоторая дилемма. С одной стороны, использование в качестве
наполнителя продукта реакции должно увеличивать чистоту выходящих
продуктов. Однако, в одной из первых работ, было показано что
использование альтернативных наполнителей (например ортофосфата кальция)
значительно увеличивает долю выхода BN в отношении к полупродуктам
\cite[]{basu_synthesis_1990}. С другой, использование альтернативных
наполнителей приводит к появлению дополнительных операций очистки BN от
примесей. Так же оставалась проблема синтеза частиц BN контролируемой
размерности и морфологии. Если с размерностью можно было справится
постобработкой готовых порошков (размол в шаровой мельнице), то синтез 
новых структурных модификаций требовал пересмотра
традиционного подхода.

Одним из первых промышленных процессов синтеза гексагонального нитрида бора является
процесс О'Конора \cite[]{edmond_process_1966}. Данный метод основан на взаимодействии 
борной кислоты и карбамида:

\reaction{(NH2)2CO + 2H3BO3 -> 2BN + CO2 + 5H2O}

Исходную смесь карбамида и нитрида бора спекают при температуре 
\SIrange{200}{300}{\degreeCelsius}, далее спек измельчают термообрабатывают в протоке 
азота и аммиака при температуре \SIrange{800}{1200}{\degreeCelsius}. Данный метод
широко применяется в синтезе микро- и нанопорошков \ce{h-BN}. Основным недостатком
данного метода является высокая турбостратность структуры целевого продукта.
Использование данного метода не позволяет получить материал контролируемой 
графитоподобной структуры, а так же материал, полученный данным методом 
обладает высоким индексом графитации (<<g>> $>$ 4).  


В 2005 году был зарегистрирован патент \cite[]{_ru__2005} развивающий данный 
подход к синтезу гексагонального нитрида бора. Группа исследователей под
руководством А.~С.~Нечепуренко разработали метод синтеза гексагонального нитрид
бора, позволяющий получить материал однородной структуры с контролируемо-низким
индексом графитации (<<g>> \ge 1.7). Ключевой особенностью прилагаемого метода
является предварительное деление исходной шихты \ce{(NH2)2CO + 2H3BO3} на две части 
различных по соотношению карбамида к борной кислоте, а так же их предварительной
термической обработки до \SI{400}{\degreeCelsius}. При температуре \SI{400}{\degreeCelsius}
спеки смешивают и размалывают в соотношении, которое соответствует необходимому 
индексу графитации конечного продукта, после чего спек азотируют в среде \ce{N2}
при температуре \SI{1850}{\degreeCelsius}.

Полученный нитрид бора имеет чистоту \SI{98.8}{\text{масс.}\%} и примесью по 
борному ангидриту около \SI{0.12}{\text{масс.}\%}.

\subsection{Метод самораспространяющегося высокотемпературного синтеза}%
\label{sub:Метод самораспространяющегося высокотемпературного синтеза}

Большим прорывом в синтезе нитридных керамик стало открытие процессов
самораспространяющегося высокотемпературного синтеза (СВС). Ведущую роль
в открытии и развитии методов СВС сыграла группа советских
исследователей: И.~П.~Боровинская, В.~М.~Ширко и А.~Г.~Мержанов. Достижения
в области СВС позволили пересмотреть промышленный подход к синтезу нитрида
бора. Ключевой особенность СВС является синтез конечного продукта благодаря
энергии выделяющейся в ходе экзотермической реакции. Использование методов 
СВС для синтеза нитридных керамик привело к существенной экономии на затрачиваемой
энергии.

Графитоподобный нитрид бора можно получить горением аморфного бора под давлением 
азота с большим выделением энергии \cite[]{loryan_nanosized_2009}:

\reaction[che:shs_BN]{2B + N2 -> 2BN + Q}

Из аморфного бора прессуют объемную заготовку \SI{0.21}{\gram\per\cm^3} в среде
азота. Далее под давлением \SIrange{10}{150}{\mega\pascal} совершается
инициирование реакции. При увеличении давления \ce{N2}  уменьшается количество
исходного аморфного бора в продуктах реакции вплоть до 0.1 масс.\% при давлении 
\SI{100}{\mega\pascal}. Температура горения достигает \SIrange{2473}{2523}{\kelvin}.

Однако, данный способ имеет значимое технологическое ограничение: требование 
по наличию азотной атмосферы под столь высоким давлением может привести к значительным сложностям
с точки зрения промышленной безопасности и технологичности производства.

Альтернативный способ синтеза порошка гексагонального нитрида бора был запатентован 
группой И.~П.~Боровинской в 1999~в. \cite[]{__1999}. Данный патент основывается на 
методе восстановления оксида бора в присутствии магния
в атмосфере азота:

\reaction[chem:bn_shs_mg]{B2O3 + 3Mg + N2 -> 2BN + 3MgO + Q}

Синтез готового порошка проводится в несколько этапов: подготовка реакционной смеси,
запресовка, инициирование реакции, забор и размол продуктов реакции, чистка от примесей,
сушка. Давление азота в камере составляет \SIrange{8}{10}{\mega\pascal}. Очистка 
целевого продукта проводится путем обработки продуктов синтеза 
\SIrange{10}{26}{\%}-серной кислотой.  

Рекомендуемое соотношение реагентов: \ce{B2O3}~48~масс.\%, \ce{Mg}~32~масс.\%., 
а так же смесь реагентов в соотношении 30~масс.\%~\ce{\text{h-}BN} и 
70~масс.\%~\ce{MgO}. Так же, продуктами реакции покрывается графитовая подложка
реактора, а так же боковые стенки. Реакция протекает в течение 7 минут и при
температуре \SI{1800}{\degreeCelsius}.  

Продукты реакции добавляются в исходную смесь с целью уменьшения скорости реакции, 
а также уменьшения температуры горения. В сравнении с методом~\ref{che:shs_BN} 
реакция протекает при значительно более низких температурах. Благодаря более низкой
температуре в продуктах реакции отсутствуют следы плавления \ce{BN}, а так же 
исключается образование сложноудаляемых боридов магния. После отмывки продуктов
в \SI{20}{\%}-ной серной кислоте при температуре \SI{100}{\degreeCelsius}
и последующей промывки водой целевой продукт содержит гексагональный нитрид бора.
Выход готовой продукции составляет \SI{22}{\%}, что соответствует \SI{98}{\%}  
относительно исходного \ce{B2O3}. 

Как можно увидеть, способ получения графитоподобного нитрида бора,
запатентованный Боровинской \cite[]{__1999}, отличается большей технологичностью
и экономичностью в связи меньшими затратами \ce{N2}, а так же отсутствию необходимости
постоянного перемешивания и размола полупродуктов.

Другим распространенным методом получения является азидная технология СВС.
В 1970 году был запатентован новый способ получения нитридных керамик с помощью
азидов - неорганических соединений солей азотистоводородной кислоты \cite[]{moss_glossary_1995}.
Авторами изобретения стали Мержанов~А.Г., Косолапов~В.~Т., Шмельников~В.~В. и
Левашев~А.~Ф. В результате проведенной работы была разработана новая марка
порошков СВС-Аз \cite{amosov2007azidnaa}. Работы по разработке данного метода 
и улучшению технико-физических свойств порошков велись на базе КПтИ, ныне Самарского
государственного технического университета (СамГУ) с 1962 г. по текущий день.
Новая методика позволила обойти ограничения, свойственные для СВС технологии.
Производство порошков тугоплавких соединений неизбежно связан с компромиссами:
комбинация из высокой степени чистоты, контролируемой структуры, заданного
гранулометрического состава, отсутствием спеков порошка часто идут в разрез друг
другу или производительности процесса в общем. Ведущую роль в развитии данного 
направления сыграл докт.техн.наук.~-~Бичуров~Г.~В. \cite[]{__2003}. 
Использование азидов в качестве источников азота в паре с галогенидами обладает
следующими преимуществами:

\begin{itemize}
    \item Использование в процессе синтеза твердых азотсодержащих элементом приводит
    увеличению концентрации реагирующих веществ в зоне реакции, что приводит к 
    более полному протеканию реакции, а диффузия газообразного реакционного газа
    перестает быть лимитирующим фактором.
    \item Переход твердого вещества в газообразный азот приводит к разрыхлению навески
    и убирает необходимость ее постоянного перемешивания.
    \item Низкая температура протекания реакции затрудняет рекристаллизацию продуктов,
    плавление полупродуктов, агломерацию частиц, что в свою очередь позволяет
    получать порошок близкий по размерности к азотируемому веществу.
\end{itemize}

Таким образом отличительной особенностью данного метода является возможность
получения высококачественного низкоразмерного порошка нитридов.

В работе \cite{amosov2007azidnaa} был проведен термодинамический анализ~\ref{fig:BN-SHS} следующих
реакций образования BN:

\reaction{4B + NaN_3 + Na4Cl -> 4 BN + NaCl + 2H_2}
\reaction{4B + NaN_3 + NH_4F -> 4 BN + NaF + 2H_2}
\reaction{4B + 3NaN_3 + KBF_4 -> 9 BN + 3NaF + KF}
\reaction{12B + 4NaNa_2 + NH_4BF_4 -> 13 BN + NaCl + 2H_2}

Наиболее высокий выход продукта демонстрируют системы в которых присутствует 
хлорид или фторид аммония (кривые 1 и 2). Насыщение зависимостей $T_\text{ад.} = f(P)$
происходит при давлении \SI{20}{\mega\pascal}  для \ce{NH3Cl} и \SI{50}{\mega\pascal}
для \ce{NH4F}. Исходя из вышесказанного наиболее перспективной системой для промышленного
применения может являться система содержащая \ce{NH4Cl}.

\begin{figure}[ht]
    \centerfloat)){
    \includegraphics[scale=0.67]{BN-SHS}}
    \legend{1,2 – мольный выход BN в системе \ce{4B} - \ce{3NaN3} - \ce{NH4Cl}(\ce{NH4F})}
    \legend{3 – мольный выход BN в системе \ce{8B} - \ce{3NaN3} – \ce{KB4}} 
    \legend{4 – мольный выход BN в системе \ce{12B} - \ce{4NaN3} – \ce{NH4BF4}} 
    \legend{5 – адиабатическая температура горения в системе \ce{12B} - \ce{4NaN3} – \ce{NH4BF4}} 
    \legend{6 – адиабатическая температура горения в системе \ce{8B} - \ce{3NaN3} – \ce{KBF4}} 
    \legend{7 – адиабатическая температура горения в системе \ce{4B} - \ce{3NaN3} - \ce{NH4Cl}}
    \legend{8 – адиабатическая температура горения в системе \ce{4B} - \ce{NaN3} - \ce{NH4F}}
    \caption{Зависимость адиабатической температуры горения систем~СВС}
    \label{fig:BN-SHS}
\end{figure}

Оптимальными получения нитрида бора по азидной технологии для систем 
\ce{B - NaN3 - NH4Cl} и \ce{B - NaN3 - KBF4} являются следующие условия:
 
\begin{itemize}
    \item давление азота \SI{5}{\mega\pascal};
    \item соотношение компонентов в системе стехиометрическое; 
    \item насыпная плотность шихты ($\delta=$0,35)
    \item удельная плотность порошка бора \SIrange{10}{25}{\milli\gram} 
 \end{itemize}

\subsection{Методы плазмохимического синтеза}%
\label{sub:Методы плазмохимического синтеза}

Основным ограничением технологий прямого синтеза синтеза BN наночастиц является
узкая структурная вариативность целевого продукта. Основным морфологическим типом
продуктов полученных методами СВС и прямого азотирования является порошок сферической
или неправильной формы. Однако, по аналогии с графитом, нитрид бора способен
образовывать:

\begin{itemize}
    \item Сферы 
    \item Нанотрубки
    \item Нанолисты
    \item Игольчатые структуры
    \item Наноленты
    \item Пленки
\end{itemize}

Оптимальным промышленным методом для получения \ce{h-BN} большего морфологического 
разнообразия является плазмохимический синтез. Данный способ основан на азотировании
бор содержащих соединений (в частности аморфного бора, хлорида бора и др.) в 
струе низкотемпературной азотной плазмы. Для увеличения выхода целевого продукта
в плазму может так же подаваться аммиак.

Подобная методика может обеспечить стабильный выход ультрадисперсного порошка \ce{h-BN},
однако работа с аммиаком требует специальных условий эксплуатации оборудования, а также
особых мер области безопасности и охраны труда. 

Значительных успехов в области промышленной реализации плазмохимического синтеза
добился АО <<Сибирский Химический Комбинат>>, на базе которого в 1997 году была 
разработана методика плазмохимического синтеза ультрадисперсного порошка \ce{h-BN}
из водного раствора борной кислоты и карбамида в низкотемпературной азотной плазме
в присутствии углеводорода \cite[]{__1997}.  

Карбамид и борную кислоту берут в соотношении (\SIrange{2}{5}{\mol}):\SI{1}{\mol},
в качестве углеводорода берут природный газ или пропан с \SIrange{5}{12}{\%} избытком
от стехиометрии разложения воды.

Синтез нитрида бора происходит в несколько этапов. На первом этапе происходит
разложение воды в присутствии углеводорода:

Для метана:
\reaction{H2O + CH4 -> CO + H2}

Для пропана:
\reaction{3H2O + C3H8 -> 3CO + 7H2}

Далее \ce{B2O3} взаимодействует с \ce{CO}, а восстановленный \ce{B} азотируется
плазмой.

Целевой продукт получается с чистотой до \SI{99.6}{\%}, а основными примесями являются 
борный ангидрид (до \SI{0.9}{\%}) и свободный углерод (до \SI{0.7}{\%}).

На основе данного метода модно получить следующие структуры:

\begin{itemize}
    \item Нанотрубки \cite[]{kim_boron_2018, yeh_stable_2017} 
    \item Нанолистья \cite[]{merenkov_orientation-controlled_2019}
\end{itemize}


\subsection{Химическое осаждение из газовой фазы}%
\label{sub:Химическое осаждение из газовой фазы}

Одним из самых гибких с позиции получения материалов различных структурных
модификаций является метод синтеза \ce{h-BN} из газовой фазы (CVD).
Метод заключается в испарении прекурсора с последующей газофазовой реакцией с аммиаком и
осаждением продукта реакции. Общая формула может быть записана как:

\reaction{B2O3 + 2NH3 -> 2BN + 3H2O}

Для контроля конечной морфологии продукта могут варьироваться следующие параметры:
состав прекусрсоров, давление реакционного газа, скорость потока транспортного газа и
температурный режим. Вариативное изменение данных параметров может привести к формированию
самых широко спектра мофологических вариаций нитрида бора \cite[]{kovalskii_growth_2016}.
В работе \cite[]{kovalskii_growth_2016} было показано влияние параметров синтеза
на морфологию целевого продукта (рис.~\ref{fig:BNO}).

\begin{figure}[ht]
    \centerfloat{
    \includegraphics[scale=0.20]{BNO-CVD}}
    \legend{a) - испарение прекурсора, b) - формирование тонкостенной сферы боарта,
    c) - формирование толстостенной сферы бората, d) - 
    образование тумана из крошечных капель жидкости и их столкновение, 
    приводящие к образованию твердых частиц, e) - низкоконцентрированный пар, 
    f) - столкновение расплавленных капель с частицами боратов; $T_1>T_2>T_3$;
    цвета: красные, синие, серые точки - испарение прекурсора, желтые
    - сконцентрированные капли}
    \caption{Пути синтеза наноструктур BN в CVD процессе
    }
    \label{fig:BNO}
\end{figure}

Несмотря на все многообразие структур, которые можно получить данным методом,
его главным недостатком является низкие технологичность и экономичность. Использование
аммиака, а так же требование постоянного поддержания нагрева делают данный метод
слабо пригодным для масштабирования и применения в промышленных технологиях.

Сравнительная характеристика порошка \ce{h-BN} полученного различными методами приведена
в таблице~\ref{tab:char}.

\begin{table}
\begin{threeparttable}% выравнивание подписи по границам таблицы
\caption{Сравнительная оценка качества порошков различных технологий синтеза~\cite[]{amosov2007azidnaa}}%
\label{tab:char}%
\begin{tabular}{| l | c | c | c | c |}
\hline
\multicolumn{1}{|c|}{\multirow{2}{*}{\textbf{Характеристика}}} & \multicolumn{4}{c|}{\textbf{Технология}} \\ \cline{2-5} 
\multicolumn{1}{|c|}{} & ПС & ПХС & СВС & СВС-Аз \\ \hline \hline
Содержание осн.в.-ва, \SI{}{\%} & 90-98 & 98-99 & 97-99 & 98-99 \\ \hline
Содержание \ce{N}, своб., мас.\SI{}{\%} & 48-55 & 40-53 & 53,6 & 55,2-55,6 \\ \hline
Содержание \ce{B}, своб., мас.\SI{}{\%} & 2,3-1,2 & 5,3 & 0,2 & до 0,2 \\ \hline
Содержание \ce{O}, мас.\SI{}{\%} & 1,3-0,1 & 10-1 & 0,6-0,3 & до 0,6 \\ \hline
Содержание \ce{C}, своб., мас.\SI{}{\%} & 1,5-0,5 & 4,5 & 0,8 & до 0,1 \\ \hline
Удельная поверхность, \SI{}{\metre^2\per\gram} & 6-9 & 10-40 & 6-70 & до 30 \\ \hline   
\end{tabular}
\end{threeparttable}
\end{table}

\chapter{Заключение}%
\label{cha:Заключение}

В данном реферате автор рассмотрел основные современные методы синтеза \ce{h-BN}.
Были представлены основные закономерности формирования структур нитрида бора, а 
так же сильные и слабые стороны тех или иных методов синтеза.

К основным промышленным методам синтеза можно отнести синтез по СВС технологии, 
а так же плазмотермический синтез. Именно эти два метода позволяют получить порошок
целевого продукта высокого качества с высокой эффективностью. 

Методы основанные на прямом азотировании прекурсора в печи, а так же CVD метод
имеют ряд недостатков. Метод прямого азотирования требует дополнительных операций
по гомогенизации полупродуктов, а так же весьма узкий диапазон морфологий целевого продукта.
Метод CVD напротив, обладает широкими возможностями по синтезу необходимых структур,
однако отличается низкой технологиченостью и неэффективностью.




